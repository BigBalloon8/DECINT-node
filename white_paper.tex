\documentclass[11pt]{extarticle}
\usepackage[margin=1in]{geometry}
\usepackage[parfill]{parskip}
\usepackage{amsmath}

\title{DECINT WHITE PAPER}
\author{Christopher Rae \\ \begin{small} email: raecd123@gmail.com \end{small}}

\date{}

\newcommand{\textify}[1]{\text{\begin{scriptsize}#1\end{scriptsize}}}

\begin{document}
\maketitle

\begin{abstract}
TEMP
\end{abstract}

\section{Introduction}

\section{Blocks} 
\subsection{Transactions}
Transaction on the DECINT blockchain use the SECP112r2 ecdsa encryption curve to sign transactions transaction are signed slightly differently depending on the transaction type. Currently there 3 different types of transaction the first being the normal token transfer transaction it looks like this:
\
\begin{center}
\{time: 1671020930.9900985,  sender: "...",  receiver: "...",  amount: 657.0,  sig: "..."\}
\end{center}

These types of transactions are made up of 5 parts. The time in Unix Time, the senders public key which consists of a 56 character string generated from the SECP112r2 curve, followed by the receivers public key, then the amount sent and finally the transaction signature the transaction signature is generate from a string of all the information in the transaction separated by a blank space " " and signed using the senders private key. All token transfer transactions have a 1\% transaction fee, 0.5\% of the transaction goes to the validator of the block which the transaction is in, the other 0.5\% is put towards the AI training nodes.  

The next 2 types are are pretty similar these are the stake and unstake transactions. Both have the same structure:
\
\begin{center}
\{time: 1671020930.9900985,  pub\_key: "...",  stake\_amount: 657.0,  sig: "..."\}
\end{center}

In the case of unstaking stake\_amount becomes unstake\_amount. the process of signing is the same as with token transfer transaction. There are no transaction fees associated with staking and unstaking.
\subsection{Temp Blocks}
Blocks are created based on time. As of writing every transaction within 2 minutes from the first transaction in a block is added to that block, the next transaction after that 2 minutes, acts as the first transaction in the next block and the process repeats. 

Temp blocks are made up of 3 parts the head, main body and tail. The head is a list of values the first being the hash of the previous temp block followed by the block index and the time of the first transaction in the block. 

The main body is made up of all the transactions that occur within the time allocated for that block.

 The tail is added when the next block is created, it is made up of 3 parts.The first holds the blocks hash and the time of the first transaction of the next block, the second part is the total transaction fees rewarded for validating that block and the final part has 2 values a False boolean value to indicate that the block has not been validated and the time of the first transaction of the next block.

\subsection{Chain Block}
The chain block is the most important block when a block is validated it becomes a chain block. much like the temp block the chain block also is made up of a head a main body and a tail, the head is the same as the temp block the main body however is made up of a json object where the keys are the wallets that have a value all transaction recorded in the temp block effect the values stored in the wallet so if Bob has 100 DCNT and send 20 to Alice Bobs wallet value would go down by 20 and Alice's wallet would go up by 19.8 and the other 0.2 would be split amongst the validator and the AI nodes. If Bob staked 20 of his DCNT his wallet would go down by 20. Instead of storing every transaction the nodes only store the current wallet values. 

The Tail is once again made up of 3 parts the first 2 not changing from the temp block but the 3rd part now contains 3 values the first being a True boolean value, the second being the time of validation and the third being the validators public key. The wallet value of the validator increase by the validation reward of the block within the main body of the block.



\section{Validation}

\section{Communication}
\subsection{Structure}
\subsection{Protocols}

\section{AI}
\subsection{Problems}

\bibliographystyle{plain}
\bibliography{reference}
\end{document}